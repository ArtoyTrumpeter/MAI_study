\documentclass[12pt]{article}

\usepackage{fullpage}
\usepackage{graphicx}
\usepackage{multicol,multirow}
\usepackage{tabularx}
\usepackage{ulem}
\usepackage{pgfplots}
\usepackage[utf8]{inputenc}
\usepackage[russian]{babel}

\begin{document}
\thispagestyle{empty}
\begin{center}
	\bfseries

	{\Large Московский авиационный институт\\ (национальный исследовательский университет)

	}

	\vspace{48pt}

	{\large Факультет информационных технологий и прикладной математики
	}

	\vspace{36pt}


	{\large Кафедра вычислительной математики и~программирования

	}


	\vspace{48pt}

	{Лабораторная работа №\,8 по курсу дискретного анализа: Жадные алгоритмы}

\end{center}

\vspace{72pt}

\begin{flushright}
	\begin{tabular}{rl}
		Студент:       & А.\, О. Тояков   \\
		Преподаватель: & А.\, Н. Ридли \\
		Группа:        & М8О-307Б-18      \\
		Дата:          &                 \\
		Оценка:        &                  \\
		Подпись:       &                  \\
	\end{tabular}
\end{flushright}

\vfill

\begin{center}
	\bfseries
	Москва\\
	\the\year
\end{center}

\newpage

\subsection*{Условие}

Разработать жадный алгоритм решения задачи, определяемой своим вариантом. Доказать его корректность, оценить скорость и объём затрачиваемой оперативной памяти.

Реализовать программу на языке С или С++, соответсвующую построенному алгоритму. Формат входных и выходных данных описан в варианте задания.

На координатной прямой даны несколько отрезков с координатами [Li, Ri]. Необходимо выбрать минимальное количество отрезков, которые бы полностью покрыли интервал [0, M].

\subsection*{Метод решения}

Для начала мне понадобится структура Segment, чтобы хранить отрезки. В ней будут хранится 3 числа: left, right , ind, которые будут обозначать левый конец отрезка, правый конец отрезка и индекст отрезка соответственно.

Для реализации поставленной задачи понадобится вектор. Проходя по всем отрезкам, мы будем добавлять в вектор такой отрезок, чтобы его левый конец покрывал уже пройденную область, а правый был максимальным. Будем выполнять этот алгоритм, пока не покроем всю область, иначе выведем 0 (по усл.). 

\subsection*{Описание программы}

Реализованный жадный алгоритм очень прост. В функции main мы вводим исходные значения, а затем запускаем функцию Selection, которая и решает поставленную задачу. В ней мы инициализируем вектор answer, в котором мы будем хранить отрезки, покрывающие область. Затем запускается цикл, в котором проверяется условие, что правый край крайнего отрезка в векторе меньше введённого числа m, и тогда мы проходим по всем отрезкам, находя отрезок оптимальной длины, т. е. такой, что его левый край покрывает уже пройденный путь, а правый является максимальным. Если мы не нашли такой отрезок, то выходим из функции и печатаем 0, иначе мы заносим его в вектор answer. На выходе из цикла вектор сортируется по индексам с помощью компаратора и результат печатается на выход.

\subsection*{Исходный код}

\begin{verbatim}
#include <iostream>
#include <vector>
#include <algorithm>

using namespace std;

struct TSegment {
	int left;
	int right;
	int ind;	
};

bool Cmp (TSegment &a, TSegment &b) {
	return a.ind < b.ind;
}

void Selection(vector<TSegment> &segs, int m) {
	vector<TSegment> answer;
	TSegment begin;
	begin.left = 0;
	begin.right = 0;
	answer.push_back(begin);
	while (answer.back().right < m) {
		int max = 0;
		int index = -1;
		for (int i = 0; i < segs.size(); i++) {
			if (segs[i].left <= answer.back().right && segs[i].right > answer.back().right) {
				if (segs[i].right > max) {
					max = segs[i].right;
					index = i;
				}
			}
		}
		if (index == -1) {
			cout << "0\n";
			return;
		} else {
			answer.push_back(segs[index]);
		}
	}

	sort(answer.begin() + 1, answer.end(), Cmp);
	cout << answer.size() - 1 << '\n';
	for (int i = 1; i < answer.size(); i++) {
		cout << answer[i].left << " " << answer[i].right << '\n';
	}
	return;

}

int main() {
	int n, m;
	cin >> n;
	vector<TSegment> segs(n);
	for (int i = 0; i < n; i++) {
		segs[i].ind = i;
		cin >> segs[i].left >> segs[i].right; 
	}
	cin >> m;
	Selection(segs, m);
	return 0;
}
\end{verbatim}

\subsection*{Дневник отладки}

Неправильный ответ на тесте 3: итоговый массив не сортировался, поэтому отрезки могли выводиться в неправильном порядке. Была добавлена сортировка с помощью функции std::sort и компаратора Cmp, который сортирует элементы по индексам.

\subsection*{Тест производительности}

Тесты представляют из себя файлы, в которых сгенерировано различное число отрезков, концы которых могут быть числами от -10000 до 10000.

\begin{tikzpicture}
\begin{axis}
\addplot coordinates {
( 10000, 0.0018)
( 100000, 0.0038)
( 500000, 0.016)
( 1000000, 0.033)};
\end{axis}
\end{tikzpicture}
\newline
Пояснения к графику:
Ось y - время в секундах. Ocь x - количество отрезков.

\subsection*{Выводы}

Жадные алгоритмы, как и динамическое программирование основываются на свойстве оптимальности подзадач. В жадном алгоритме мы принимаем локальные решения, допуская, что конечное решение также окажется оптимальным. Это работает не всегда и в данной лабороторной сложность меняется от O(n) до O($n^2$) в зависимости от входных данных. Однако, если данное свойство выполняется, то реализовать жадный алгоритм будет проще, чем наивный, и также мы получим выигрыш в скорости. 

Проверить это можно с помощью так называемого матроида. Он представляет из себя конечный автомат c двумя состояниями (X, I). В свою очередь для него также должны выполнятся 3 свойства, а именно: множество I не может быть пустым, любое подмножество любого элемента из I также будет элементом этого множества и если А и B - множества, принадлежащие I, причём А < B, то какой нибудь элемент x из B, не принадлежащий A, в объединении с А будет элементом множества I.

Если доказать, что наш объект это матроид, то жадный алгоритм будет работать корректно, иначе придётся пользоваться, например, динамическим программированием или даже наивными алгоритмами.

\end{document}

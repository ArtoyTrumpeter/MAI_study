\documentclass[12pt]{article}

\usepackage{fullpage}
\usepackage{graphicx}
\graphicspath{{pictures/}}
\usepackage{multicol,multirow}
\usepackage{tabularx}
\usepackage{ulem}
\usepackage{pgfplots}
\usepackage[utf8]{inputenc}
\usepackage[russian]{babel}

\begin{document}
\thispagestyle{empty}
\begin{center}
	\bfseries

	{\Large Московский авиационный институт\\ (национальный исследовательский университет)

	}

	\vspace{48pt}

	{\large Факультет информационных технологий и прикладной математики
	}

	\vspace{36pt}


	{\large Кафедра вычислительной математики и~программирования

	}


	\vspace{48pt}

	{Курсовой проект по курсу "Дискретный анализ" на тему \\ "Эвристический поиск в графе"}

\end{center}

\vspace{72pt}

\begin{flushright}
	\begin{tabular}{rl}
		Студент:       & А.\, О. Тояков   \\
		Преподаватель: & А.\, Н. Ридли \\
		Группа:        & М8О-307Б-18      \\
		Дата:          &                 \\
		Оценка:        &                  \\
		Подпись:       &                  \\
	\end{tabular}
\end{flushright}

\vfill

\begin{center}
	\bfseries
	Москва\\
	\the\year
\end{center}

\newpage

\subsection*{Условие}

Разработать программу на языке С или С++, реализующую указанный алгоритм согласно заданию.
\\
Задана карта, состоящая из n*m клеток. Необходимо найти путь из вершины с номером start в вершину с номером finish при помощи алгоритма $A^{*}$. Также на карте возможно наличие непроходимых точек.

\subsection*{Метод решения}

Данное задание предполагает реализацию алгоритма $A^{*}$ по нахождению пути.

\begin{enumerate}
    \item Добавить стартовую клетку в открытый список (при этому её значения G, H и F равны 0).
    \item Повторять следующие шаги:\\
    Ищем в открытом списке клетку с наименьшим значением величины F, делаем её текущей.\\
    Удаляем текущую клетку из открытого списка и помещаем в закрытый список.\\
    Для каждой из соседних к текущей точке клеток:\\
    Если клетка непроходима или находится в закрытом списке, игнорируем её.\\
    Если клетка не в открытом списке, то добавляем её в в открытый список, при этом рассчитываем для неё значения G, H и F, и также устанавливаем ссылку родителя на текущую клетку.\\
    Если клетка находится в открытом списке, то сравниваем её значение G со значением G таким, что если бы к ней пришли через соседнюю клетку. Если сохранённое в проверяемой клетке значение G больше нового, то меняем её значение G на новое, пересчитываем её значение F и изменяем указатель на родителя так, чтобы она указывала на текущую клетку.\\
    Останавливаемся, если:\\
    В открытый список добавили целевюу клетку (в этом случае путь найден). Открытый список пуст (в этом случае к целевой клетке пути не существует).
    \item Сохраняем путь, двигаясь назад от целевой точки, проходя по указателям на родителей до тех пор, пока не дойдём до стартовой клетки.
\end{enumerate}

\subsection*{Описание программы}

Мой проект состоит из трёх файлов: main.cpp, map.hpp, search.hpp. Поиск происходит на карте с закрытыми точками. В качестве эвристики было выбрано манхэттенское расстояние, а перемещение происходит по 4 направлениям.\\
map.hpp: класс map представляет собой карту и содержит один основной метод: Construct SymbolsMap(unsigned int NoPointAmount). В качестве параметра указывается количество добавлячемых непроходимых точек, которые неолбходимо ввести. Непроходимые точки добавляются в закрытый список. Соответственно карта представляет собой двумерный массив, содержащий информацию о точке: вес, значение F = вес + эвристика, координаты родителя.\\
search.hpp: данный класс представляет собой реализацию эвристичекого поиска. Октрытый список реализован с помощью приоритетной очереди, отсортированной по возрастанию. Также используется дополнительная структура для хранения актуального списка точек в открытом списке. Данный класс также содержит объект map. Соответственно соседние точки от текущей добавляются в вектор. После данные точки анализируются согласно алгоритму. Ответ формируется в виде карты, где посещённые точки отмечаются восклицательным знаком. Например:
\\
путь (4 направления и эвристика - манхэттенское расстояние)\\
стартовая точка: (2,2), конечная точка: (9,9)\\
\\
$\sharp$ $\sharp$ $\sharp$ $\sharp$ $\sharp$ $\sharp$ $\sharp$ $\sharp$ $\sharp$ $\sharp$ $\sharp$\\
$\sharp$ 0 ! ! ! 0 0 0 0 0 $\sharp$\\
$\sharp$ 0 ! $\sharp$ ! 0 0 0 0 0 $\sharp$\\
$\sharp$ 0 $\sharp$ 0 ! 0 0 0 0 0 $\sharp$\\
$\sharp$ 0 0 0 ! 0 0 0 0 0 $\sharp$\\
$\sharp$ 0 0 0 ! 0 0 0 0 0 $\sharp$\\
$\sharp$ 0 $\sharp$ 0 ! 0 0 0 $\sharp$ 0 $\sharp$\\
$\sharp$ 0 $\sharp$ 0 ! 0 $\sharp$ $\sharp$ 0 0 $\sharp$\\
$\sharp$ 0 0 $\sharp$ ! ! ! 0 $\sharp$ $\sharp$ $\sharp$\\
$\sharp$ 0 0 0 $\sharp$ $\sharp$ ! ! ! ! $\sharp$\\
$\sharp$ $\sharp$ $\sharp$ $\sharp$ $\sharp$ $\sharp$ $\sharp$ $\sharp$ $\sharp$ $\sharp$ $\sharp$\\
\\
Пояснения: закрытый список - список рассмотренных вершин. Открытый список - список вершин, которые необходимо рассмотреть.

\subsection*{Исходный код}

// main.cpp
\begin{verbatim}
#include "source/map.hpp"
#include "source/search.hpp"


int main() {
    unsigned int n,m;
    std::cin >> n >> m;
    Map* testMap = new Map(n,m);
    unsigned int amount;
    std::cin >> amount;
    testMap->ConstructSymbolsMap(amount);
    Search* testSearch = new Search(testMap);
    std::cout << "Start(x,y) && Finish(x,y)" << std::endl;
    unsigned int x,y,x1,y1;
    std::cin >> x >> y >> x1 >> y1;
    auto start = std::make_pair(x,y);
    auto end = std::make_pair(x1,y1);
    testSearch->Result(start, end);
    testMap->Print();
    delete testMap;
    delete testSearch;
    return 0;
}
\end{verbatim}
\\
//map.hpp
\begin{verbatim}
#ifndef MAP_HPP
#define MAP_HPP
#include <iostream>
#include <vector>
#include <cmath>
#include <set>
#include <queue>


struct Information {
    std::string symbol; // 0 - true, # - false
    unsigned int allScore; // weight + heuristic(finish);
    unsigned int weight;
    std::pair<unsigned int, unsigned int> from; // parent_coordinate


    Information() {
        symbol = "0";
        allScore = 0;
        weight = 0;
        from = std::make_pair(0,0);
    }
};


class Map {
private:
    unsigned int lines;
    unsigned int colums;
    std::vector<std::vector<Information>> symbolsMap;
    std::set<std::pair<unsigned int, unsigned int>> noPoints; //new # in symbolsMap

public:
    friend class Search;

    Map(unsigned int lines, unsigned int colums) {
        this->lines = lines + 2;
        this->colums = colums + 2;
        symbolsMap.resize(this->lines);
    }


    void Print() {
        for(auto i = 0; i < lines;i++) {
            for(auto j = 0; j < colums;j++) {
                std::cout << symbolsMap[i][j].symbol << " ";
            }
            std::cout << std::endl;
        }
    }


    bool CheckNewPoint(int x, int y) {
    if(x > lines - 1 || y > colums - 1 || x <= 0 || y <= 0) {
        return false;
    }
    return true;
    }


    void ConstructSymbolsMap(unsigned int NoPointAmount) {
        std::string status;
        int temp = 0;
        for(auto i = 0; i < lines; i++) {
            for(auto j = 0; j < colums; j++) {
                symbolsMap[i].push_back(Information());
            }
        }
        for(auto j = 0; j < colums;j++) {
            symbolsMap[0][j].symbol = '#';
            symbolsMap[lines - 1][j].symbol = '#';
            noPoints.insert(std::make_pair(0, j));
            noPoints.insert(std::make_pair(lines - 1, j));
        }
        for(auto i = 0; i < lines;i++) {
            symbolsMap[i][0].symbol = '#';
            symbolsMap[i][colums - 1].symbol = '#';
            noPoints.insert(std::make_pair(i, 0));
            noPoints.insert(std::make_pair(i, colums - 1));
        }
        // add new #(noPoint)
        while(temp < NoPointAmount) {
            unsigned int pointX, pointY;
            std::cin >> pointX >> pointY;
            if(CheckNewPoint(pointX, pointY)) {
                symbolsMap[pointX][pointY].symbol = '#';
                noPoints.insert(std::make_pair(pointX, pointY));
                std::cout << "ADD" << std::endl;
            }
            temp++;
        }
        /*
        Example:
            map(3,3) and noPoint{(2,2)}
            # # # # #
            # 0 0 0 #
            # 0 # 0 #
            # 0 0 0 #
            # # # # # 
            0 -  =>
            Information() {
                symbol = "0";
                allScore = 0;
                weight = 0;
                from = std::make_pair(0,0); 
            # -> # coordinate in "closedList" noPoints
            }       
        */
        Print();
    }
};


#endif
\end{verbatim}
\\
//seacrh.hpp
\begin{verbatim}
    #ifndef SEARCH_HPP
#define SEARCH_HPP
#include "map.hpp"
#include <map>
#include <algorithm>
#define const_weight 5 
using point = std::pair<unsigned int, unsigned int>;
using pointAndScore = std::pair<point, unsigned int>;


struct MyCompare {
    constexpr bool operator()(std::pair<point, unsigned int> const & a, std::pair<point, unsigned int> const & b) const noexcept {
        return a.second >= b.second;
    }
};


class Search {
public:
    Search(Map* map) {
        this->map = map;
    }

    void Result(point start, point finish) {
        if(map->CheckNewPoint(start.first,start.second) == true
            && map->CheckNewPoint(finish.first, finish.second) == true
            && AStar(start,finish) == true) {
                std::cout << "YES" << std::endl;
            ChangeSymbolMapAndPrintResult(start,finish);
        } else {
            std::cout << "ERROR" << std::endl;
        }
    }

private:
    Map* map;
    // открытый список
    std::priority_queue <pointAndScore,
            std::vector <pointAndScore>,MyCompare> openQueue;
    std::set<point> pointsInOpenList;

    void ChangeSymbolMapAndPrintResult(point start, point finish) {
        point current = finish;
        int i = 1;
        while(current != start) {
            //map->symbolsMap[current.first][current.second].symbol = std::to_string(i);
            map->symbolsMap[current.first][current.second].symbol = '!';
            current = map->symbolsMap[current.first][current.second].from;
            i++;
        }
        //map->symbolsMap[start.first][start.second].symbol = std::to_string(i);
        map->symbolsMap[start.first][start.second].symbol = '!';
    }


    bool FindElement(std::set<point>& pointsInOpenList, point& old) {
        auto it = pointsInOpenList.find(old);
        if(it == pointsInOpenList.end()) {
            return false;
        }
        return true;
    }


    void Erase(std::set<point>& pointsInOpenList, point& old) {
        for(auto it = pointsInOpenList.begin();it != pointsInOpenList.end();it++) {
            if((*it) == old) {
                pointsInOpenList.erase(it);
                return;
            }
        }
    }


    unsigned int Heuristik(point start, point finish) {
        //return (std::max(abs(start.first - finish.first),abs(start.second - finish.second)) * const_weight);
        //return (sqrt((pow((start.first - finish.first),2) + pow((start.second - finish.second),2))) * const_weight);
        return ((abs(start.first - finish.first) + abs(start.second - finish.second)) * const_weight);
    }


    void GetFourVertex(std::vector<point>& adjacentVertex,point& current) {
        int i_const,j_const;
        i_const = current.first;
        j_const = current.second;
        int i_f,j_f,i_s,j_s;
        j_f = current.second - 1;
        j_s = current.second + 1;
        adjacentVertex.push_back({i_const,j_f});
        adjacentVertex.push_back({i_const,j_s});
        i_f = i_const - 1;
        i_s = i_const + 1;
        adjacentVertex.push_back({i_f, j_const});
        adjacentVertex.push_back({i_s, j_const});
    }


    void GetAllVertex(std::vector<point>& adjacentVertex,point& current) {
        for(auto j = current.second - 1; j <= current.second + 1; j++) {
            if(j == current.second) {
                unsigned int first = j - 1;
                unsigned int second = j + 1;
                adjacentVertex.push_back({current.first, first});
                adjacentVertex.push_back({current.first, second});
            }
            unsigned int high, low;
            high = current.first + 1;
            low = current.first - 1;
            adjacentVertex.push_back({low, j});
            adjacentVertex.push_back({high, j});
        }
    }


    bool AStar(point start, point finish) {
        openQueue.push({start,0});
        pointsInOpenList.insert(start);
        while(openQueue.size() > 0) {
            pointAndScore current = openQueue.top();
            if(current.first == finish) {
                return true;
            }
            openQueue.pop();
            Erase(pointsInOpenList, current.first);
            map->noPoints.insert(current.first);
            std::vector<point> adjacentVertex;
            unsigned int tenativeScore = 
                map->symbolsMap[current.first.first][current.first.second].weight + const_weight;
            GetFourVertex(adjacentVertex,current.first);
            //GetAllVertex(adjacentVertex,current.first);
            for(size_t j = 0;j < adjacentVertex.size(); j++) {
                //Если клетка непроходима или находится в закрытом списке, игнорируем её.
                if(map->noPoints.find(adjacentVertex[j]) != map->noPoints.end()) {
                    continue;
                }
                if(FindElement(pointsInOpenList,adjacentVertex[j])) {
                    if(tenativeScore < map->symbolsMap[adjacentVertex[j].first][adjacentVertex[j].second].weight) {
                        map->symbolsMap[adjacentVertex[j].first][adjacentVertex[j].second].from = current.first;
                        map->symbolsMap[adjacentVertex[j].first][adjacentVertex[j].second].weight = tenativeScore;
                        map->symbolsMap[adjacentVertex[j].first][adjacentVertex[j].second].allScore = tenativeScore + Heuristik(adjacentVertex[j],finish); 
                    }
                } else {
                    map->symbolsMap[adjacentVertex[j].first][adjacentVertex[j].second].from = current.first;
                    map->symbolsMap[adjacentVertex[j].first][adjacentVertex[j].second].weight = tenativeScore;
                    map->symbolsMap[adjacentVertex[j].first][adjacentVertex[j].second].allScore = tenativeScore + Heuristik(adjacentVertex[j],finish);
                    //std::cout << adjacentVertex[j].first << " " << adjacentVertex[j].second << " " << map->symbolsMap[adjacentVertex[j].first][adjacentVertex[j].second].allScore << std::endl;
                    openQueue.push({adjacentVertex[j],map->symbolsMap[adjacentVertex[j].first][adjacentVertex[j].second].allScore});
                    pointsInOpenList.insert({adjacentVertex[j].first,adjacentVertex[j].second});
                }
            }
        }
        return false; 
    }
};

#endif
\end{verbatim}

\subsection*{Недочёты}

Использование менее оптимальных структур данных для хранения информации.\\
Использование дополнительного множества для хранения актуального списка точек открытого списка.

\subsection*{Пример работы программы}

Для проверки результата работы программы было составлено 5 тестов вручую и написан test-gen, генерирующий рандомные корректные входные данные.

\includegraphics[width=25cm]{screen1.png}

\includegraphics[width=25cm]{screen2.png}

\includegraphics[width=25cm]{screen3.png}

\includegraphics[width=25cm]{screen4.png}

\subsection*{Выводы}

Основное приминение данного алгоритма это нахождения минимальных путей в графе. От жадного алгоритма, который тоже является алгоритмом поиска по первому лучшему совпадению, его отличает то, что при выборе вершины он учитывает, помимо прочего, весь пройденный до неё путь. Таким образом алгоритм $A^{*}$ в нектором роде является наследником алгоритма Дейкстры. $A^{*}$ использует эвристику, с помощью которой можно отсекать некоторые неоптимальные пути. Поведение алгоритма сильно зависит от того, какая эвристика используется. В свою очередь, выбор эвристики зависит от постановки задачи. Манхэттенское расстояние используется, когда мы можем перемещаться в четырёх направлениях. Также существуют эвристики для случаев, когда к перемещению добавляются диагонали (расстояние Чебышева) или передвижение вообще не ограничено сеткой (евклидово расстояние). В худшем случае, число вершин исследуемых алгоритмом растёт экспоненциально по сравнению с длиной оптимального пути, но сложность становится полиномиальной, когда оценка h(x) не растёт быстрее, чем логарифм оптимальной эвристики.
\\

Этот алгоритм также можно улучшить некоторыми способами в виду того, что мы используем очень много памяти для хранения и выполняем некоторые необязательные действия. Во-первых, конечно, нужно подбирать оптимальную эвристику под каждую конкретную задачу. Во-вторых можно использовать такие улучшения, как ограничение объёма памяти или итеративный поиск в глубину. Это обеспечит нам быстродействие алгоритма, однако, таким образом, мы можем не найти оптимальный путь при слишком строгих ограничениях. В-третьих, нужно подбирать оптимальные структуры данных для хранения информации, например красно-чёрное дерево и булевую хэш таблицу для закрытого списка.
\\

В целом применять этот алгоритм мы может во многих жизненных задачах. Например, поиск оптимального пути к нужной точке на карте. Также $A^{*}$ широко используется в играх. В жанре Tower Defense мы можем показывать вражескому AI кратчайший путь до наших укреплений. Помимо этого $A^{*}$ использовался в очень популярной игре Warcraft 3 для тех же целей. В итоге этот алгоритм будет эффективен в тех случаях, когда нам известна конечная точка.

\end{document}
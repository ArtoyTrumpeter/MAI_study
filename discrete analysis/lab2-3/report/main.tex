\documentclass[12pt]{article}

\usepackage{fullpage}
\usepackage{graphicx}
\usepackage{multicol,multirow}
\usepackage{tabularx}
\usepackage{ulem}
\usepackage{pgfplots}
\usepackage[utf8]{inputenc}
\usepackage[russian]{babel}

\begin{document}
\thispagestyle{empty}
\begin{center}
	\bfseries

	{\Large Московский авиационный институт\\ (национальный исследовательский университет)

	}

	\vspace{48pt}

	{\large Факультет информационных технологий и прикладной математики
	}

	\vspace{36pt}


	{\large Кафедра вычислительной математики и~программирования

	}


	\vspace{48pt}

	{Лабораторная работа №\,2-3 по курсу дискретного анализа: Словарь}

\end{center}

\vspace{72pt}

\begin{flushright}
	\begin{tabular}{rl}
		Студент:       & А.\, О. Тояков   \\
		Преподаватель: & Н.\, А. Зацепин \\
		Группа:        & М8О-307Б-18      \\
		Дата:          &                 \\
		Оценка:        &                  \\
		Подпись:       &                  \\
	\end{tabular}
\end{flushright}

\vfill

\begin{center}
	\bfseries
	Москва\\
	\the\year
\end{center}

\newpage

\subsection*{Условие}

\begin{enumerate}
\item Необходимо создать программную библиотеку, реализующую указанную структуру данных, на основе которой разработать программу-словарь. В словаре каждому ключу, представляющему из себя регистронезависимую последовательность букв английского алфавита длиной не более 256 символов, поставлен в соответствие некоторый номер, от 0 до $2^{64} - 1$. Разным словам может быть поставлен в соответствие один и тот же номер.
\item Красно-чёрное дерево. Тип ключа: последовательность букв английского алфавита длиной не более 256 символов. Тип значения: числа от 0 до $2^{64} - 1$.
\end{enumerate}

\subsection*{Метод решения}

В данной работе нужно было реализовать 5 операций: поиск в дереве, удаление элемента из дерева, добавление элемента в дерево, загрузка структуры  из бинарного файла, сохранение в бинарный файл.

\begin{enumerate}
\item Поиск в дереве: поиск осуществляется так же, как и в BST, то есть мы сравниваем значения по ключу, начиная с корня и идём в левого сына, если искомый ключ меньше данного или в правого сына иначе, пока ключи не совпадут или мы не дойдём до листа (Nil вершина).
\item Добавление в дерево: для начала ищется позиция для вставки алгоритмом, который представлен выше. После этого мы присваиваем Nil вершины листам, не забываем про указатель на родителя и присваиваем красный цвет новой вершине. После этого запускается функция коррекции вставки. В ней для начала проверяется является ли наша вершина корнем. Если да то она перекрашивается в чёрный цвет (корень всегда чёрный). Затем мы смотрим на родителя вершины. Если он чёрный, то ни одно правило не нарушается и мы выходим из алгоритма. После этого стоит условие проверки цвета дяди, то есть брата отца нашей вершины. Если он красный то мы перекрашиваем его и нашего отца в чёрный цвет, а дедушку в красный и запускаем эту функцию рекурсивно от дедушки. Если дядя не оказался красным, то всё исправляется с помощью максимум двух поворотов и конечного перекрашивания.
\item Удаление из дерева: Если удаляется красная вершина, то удаление просиходит по обычным правилам удаления вершины из BST. Если удаляется чёрная вершина, то мы таким же образом удаляем её из дерева и запускаем fix-функцию, где может быть 4 случая: 1) мы берём брата fix-элемента и если он красный, то делаем его чёрным, родителя красным и делаем левый поворот относительно родителя, а затем переприсаиваем брата. 2) Если у брата оба сына чёрные, то перекрашиваем его в чёрный и переходим к родителю. 3) Если у брата правый сын чёрный, то делаем его второго сына чёрным а самого красным и делаем правый поворот относительно брата и переприсваиваем брата. 4) В последнем случае перекрашиваем брата в цвет родителя, родителя в чёрный, а правого сына брата в чёрный и делаем левый поворот относительно родителя. В конце алгоритма переприсваиваем fix-элемент на корень и делаем его чёрным. Так проходим по вершине циклично пока она не станет корнем или не станет чёрной. Итого максимум может быть 3 поворота. Выше описан случай, когда fix-вершина является левым сыном родителя. Также есть зеркальный случай, когда она правый сын.
\item Загрузка и сохранение дерева: сохраняю из загружаю соответственно в бинарный файл. Первый аргумент наш файл. Второй - корень при сохранении и Nil вершина при загрузке. При сохранении все данные для каждой вершины поочереди записываются в файл. Перемещение между вершинами осущетсвлено так: создаются две булевые переменные, которые принимают true и false в зависимости от того есть ли у нашей вершины левый и правый сын. При значении true рекурсивно запускается функция сохранения иначе выход. При загрузке сначала очищается всё текущее дерево и затем таким же образом, как при сохранении вершины записываются в дерево. Функция возвращает вершину n, которая по итогу окажется корнем нашего красно-чёрного дерева.
\end{enumerate}

\subsection*{Описание программы}

Программа реализована одним файлом. В первую очередь хотелось бы описать параметры класса TRBTree. Этот класс шаблонный, где параметры шаблона это ключ K и значение V. Параметры класса: корневой узел Root, Nil узел (по сути общий лист, указатели всех листьев будут присвоены ему) и количетсво узлов. Что касается узла, то он представляет из себя структуру, где есть ключ типа K, значение типа V, enum переменная цвета R (red) или B (black) и три указателя: на родителя, на левого сына и на правого сына. Для узла определён конструктор. Что касается самого дерева, то в нём реализованы конструктор, деструктор, 5 основных функций, а также вспомогательные. Вспомогательными функциями являются: нахождение дяди, fix вставки, fix удаления, замещение элемента другим, левый и правый повороты, а также несколько функций словаря таких, как сохранение в файл, загрузка из файла, поиск по словарю, добавление и удаление элемента в ловарь. Что касается основных функций, то примерные алгоритмы описаны выше.

\subsection*{Исходный код}
\begin{verbatim}
#include <iostream>
#include <cstring>
#include <fstream>
#include <algorithm>

const size_t BUF_SIZE = 257;

class TString {
    char* Str;
    size_t Size;
    using iterator = char *;
public:
    TString() : Str(nullptr), Size(0) {}

    TString(const char* s) {
        Size = strlen(s);
        Str = new char[Size + 1];
        std::copy(s, s + Size, Str);
        Str[Size] = '\0';
    }

    TString(const TString& s) {
        Size = s.Size;
        Str = new char[Size + 1];
        std::copy(s.Str, s.Str + Size, Str);
        Str[Size] = '\0';
    }

    char* String() {
        return Str;
    }

    void Move(char* s) {
		delete[] Str;
		Str = s;
		Size = strlen(s);
	}

    iterator begin() {
        return Str;
    }

    const iterator begin() const {
        return Str;
    }

    iterator end() {
        if (Str)
            return Str + Size;
        return nullptr;
    }

    const iterator end() const {
        if (Str)
            return Str + Size;
        return nullptr;
    }

    size_t StrSize() const {
        return Size;
    }

    char operator[](size_t idx) {
        return Str[idx];
    }

    const char operator[](size_t idx) const {
        return Str[idx];
    }

    friend std::ostream& operator<<(std::ostream& out, const TString& s);
    friend std::istream& operator>>(std::istream& in, const TString& s);

    TString& operator= (const TString& s) {
		delete[] Str;
		Size = s.Size;
		Str = new char[Size + 1];
        std::copy(s.Str, s.Str + Size, Str);
        Str[Size] = '\0';
        return *this;
	}

	TString& operator= (const char* s) {
		delete[] Str;
		Size = strlen(s);
        Str = new char[Size + 1];
        std::copy(s, s + Size, Str);
        Str[Size] = '\0';
		return *this;
	}

    ~TString() {
        delete[] Str;
        Str = nullptr;
        Size = 0;
    }
};

bool operator<(const TString& comp1, const TString& comp2) {
    size_t minSize;
    comp1.StrSize() < comp2.StrSize() ? (minSize = comp1.StrSize()) : (minSize = comp2.StrSize());
    for (size_t i = 0; i < minSize; i++) {
        if (comp1[i] != comp2[i])
            return (comp1[i] < comp2[i]);
    }
    return comp1.StrSize() < comp2.StrSize();
}

bool operator==(const TString& comp1, const TString& comp2) {
    return !(comp1 < comp2) && !(comp2 < comp1);
}

bool operator!=(const TString& comp1, const TString& comp2) {
    return !(comp1 == comp2);
}

std::ostream& operator<<(std::ostream& out, const TString& s) {
    for (auto ch : s)
        out << ch;
    return out;
}

std::istream& operator>>(std::istream& in, TString& s) {
    char buf[BUF_SIZE];
    if (in >> buf)
        s = buf;
    return in;
}

using std::cout;
using std::cin;
using std::endl;

template <typename K, typename V>
class TRBTree {
    enum TColor {R, B};

    struct TNode {
        TColor Color;
        K Key;
        V Value;
        TNode* P;
        TNode* Left;
        TNode* Right;

        TNode(const K& nodeKey, const V& nodeVal) :
        Color(R), Key(nodeKey), Value(nodeVal), P(nullptr), Left(nullptr), Right(nullptr) {}

        TNode() : P(nullptr), Left(nullptr), Right(nullptr) {}
    };
    
    TNode* Root;
    TNode* Nil;
    size_t Size;

    TNode* Uncle(TNode* n) {
        if (n->P == Nil) {
            return Nil;
        }
        if (n->P->P == Nil) {
            return Nil;
        }
        if (n->P->P->Right == n->P) {
            return n->P->P->Left;
        }
        return n->P->P->Right;
    }

public:
    TRBTree() {
        Nil = new TNode;
        Root = Nil;
        Root->Color = B;
    }
    TRBTree(const K& rootKey, const V& rootVal) {
        Nil = new TNode;
        Nil->Color = B;
        Root = new TNode(rootKey, rootVal);
        Root->Color = B;
        Root->P = Nil;
        Root->Left = Nil;
        Root->Right = Nil;
        Size++;
    }

    void Clean(TNode* n) {
        if (n == Nil)
            return ;
        Clean(n->Left);
        Clean(n->Right);
        delete n;
    }

    ~TRBTree() {
        Clean(Root);
        delete Nil;
        Size = 0;
    }

    void Insert(const K& key, const V& val) {
        TNode* n = new TNode(key, val);
        TNode* tmp = Root;
        TNode* p = tmp;
        if (tmp != Nil) {
            while (tmp != Nil) {
                p = tmp;
                if (n->Key < tmp->Key) {
                    tmp = tmp->Left;
                } else {
                    tmp = tmp->Right;
                }
            }
            if (n->Key < p->Key) {
                p->Left = n;
            } else {
                p->Right = n;
            }
        } else {
            Root = n;
        }
        n->Color = R;
        n->P = p;
        n->Left = Nil;
        n->Right = Nil;
        Size++;
        InsertFix(n);
    }

    void InsertFix(TNode* ins) {
        if (ins == Root) {
            ins->Color = B;
            return ;
        }
        if (ins->P->Color == B) {
            return ;
        }
        TNode* unc = Uncle(ins);
        if (unc->Color == R) { //1
            unc->Color = B;
            ins->P->Color = B;
            ins->P->P->Color = R;
            InsertFix(ins->P->P); //max h/2 iterations
        } else {
            if (ins == ins->P->Right && ins->P == ins->P->P->Left) { //2
                LeftRotate(ins->P);
                ins = ins->Left;
            } else if (ins == ins->P->Left && ins->P == ins->P->P->Right) { //3
                RightRotate(ins->P);
                ins = ins->Right;
            }
            if (ins == ins->P->Left) { //4
                RightRotate(ins->P->P);
                ins->P->Color = B;
                ins->P->Right->Color = R;
            } else if (ins == ins->P->Right) { //5
                LeftRotate(ins->P->P);
                ins->P->Color = B;
                ins->P->Left->Color = R;
            }
        }
    }

    void Replace(TNode* n, TNode* rep) {
        if (n == Root)
            Root = rep;
        else
            (n->P->Left == n) ? (n->P->Left = rep) : (n->P->Right = rep);
        rep->P = n->P;
    }

    void Erase(K& key) {
        TNode* n = FindNode(key);
        if (n != Nil) {
            Erase(n);
        }
    }

    void Erase(TNode* n) {
        TNode* del = n;
        TNode* fixNode = Nil;
        TColor delColor = n->Color;
        if (n->Left == Nil) {
            fixNode = n->Right;
            Replace(n, n->Right);
        } else if (n->Right == Nil) {
            fixNode = n->Left;
            Replace(n, n->Left);
        } else {
            TNode* tmp = n->Right;
            while (tmp->Left != Nil) {
                tmp = tmp->Left;
            }
            del = tmp;
            delColor = del->Color;
            fixNode = del->Right;
            if (del->P == n) {
                fixNode->P = del;
            } else {
                Replace(del, del->Right);
                del->Right = n->Right;
                del->Right->P = del;
            }
            Replace(n, del);
            del->Left = n->Left;
            del->Left->P = del;
            del->Color = n->Color; //node replacement (changing Key and Value but Color stays the same)
        }
        if (delColor == B) {
            EraseFix(fixNode);
        }
        delete n;
    }

    void EraseFix(TNode* n) {
        while (n != Root && n->Color == B) {
            if (n->P->Left == n) { //n is Left child
                TNode* bro = n->P->Right;
                if (bro->Color == R) { //1
                    bro->Color = B;
                    bro->P->Color = R;
                    LeftRotate(bro->P);
                    bro = n->P->Right;
                }
                if (bro->Left->Color == B && bro->Right->Color == B) { //2
                    bro->Color = R;
                    n = n->P;
                } else {
                    if (bro->Right->Color == B) { //3
                        bro->Left->Color = B;
                        bro->Color = R;
                        RightRotate(bro);
                        bro = n->P->Right;
                    }
                    bro->Color = bro->P->Color; //4
                    bro->P->Color = B;
                    bro->Right->Color = B;
                    LeftRotate(bro->P);
                    n = Root;
                }
            } else { //n is Right child
                TNode* bro = n->P->Left;
                if (bro->Color == R) {
                    bro->Color = B;
                    bro->P->Color = R;
                    RightRotate(bro->P);
                    bro = n->P->Left;
                }
                if (bro->Left->Color == B && bro->Right->Color == B) {
                    bro->Color = R;
                    n = n->P;
                } else {
                    if (bro->Left->Color == B) {
                        bro->Right->Color = B;
                        bro->Color = R;
                        LeftRotate(bro);
                        bro = n->P->Left;
                    }
                    bro->Color = bro->P->Color;
                    bro->P->Color = B;
                    bro->Left->Color = B;
                    RightRotate(bro->P);
                    n = Root;
                }
            }
        }
        n->Color = B;
    }

    void RightRotate(TNode* pivot) {
        TNode* tmp = pivot->Left;
        pivot->Left = tmp->Right;
        if (tmp->Right != Nil)
            tmp->Right->P = pivot;
        tmp->P = pivot->P;
        if (pivot->P == Nil) {
            Root = tmp;
        } else {
            if (pivot->P->Right == pivot) {
                pivot->P->Right = tmp;
            } else {
                pivot->P->Left = tmp;
            }
        }
        tmp->Right = pivot;
        pivot->P = tmp;
    }

    void LeftRotate(TNode* pivot) {
        TNode* tmp = pivot->Right;
        pivot->Right = tmp->Left;
        if (tmp->Left != Nil)
            tmp->Left->P = pivot;
        tmp->P = pivot->P;
        if (pivot->P == Nil) {
            Root = tmp;
        } else {
            if (pivot->P->Right == pivot) {
                pivot->P->Right = tmp;
            } else {
                pivot->P->Left = tmp;
            }
        }
        tmp->Left = pivot;
        pivot->P = tmp;
    }

    void Print() {
        PrintNode(Root, 0);
        cout << endl << endl;
    }

    void PrintNode(TNode* n, size_t shift) {
        if (n == Nil) {
            return ;
        }
        PrintNode(n->Right, shift + 7);
        for (size_t s = 0; s < shift; s++) {
            cout << " ";
        }
        cout << n->Key << ": " << n->Value << " ";
        n->Color == R ? (cout << "Red" << endl) : (cout << "Black" << endl);
        PrintNode(n->Left, shift + 7);
    }

    TNode* FindNode(K& key) {
        TNode* n = Root;
        while (n->Key != key && n != Nil) {
            if (key < n->Key) {
                n = n->Left;
            } else {
                n = n->Right;
            }
        }
        return n;
    }

//vocabulary
    void VocAdd() {
        K key;
        V val;
        cin >> key >> val;
        std::transform(key.begin(), key.end(), key.begin(), ::tolower);
        TNode* n = FindNode(key);
        if (n == Nil) {
            Insert(key, val);
            cout << "OK\n";
        } else {
            cout << "Exist\n";
        }
    }

    void VocErase() {
        K key;
        cin >> key;
        std::transform(key.begin(), key.end(), key.begin(), ::tolower);
        TNode* n = FindNode(key);
        if (n != Nil) {
            Erase(n);
            cout << "OK\n";
        } else {
            cout << "NoSuchWord\n";
        }
    }

    void VocFind(K& key) {
        std::transform(key.begin(), key.end(), key.begin(), ::tolower);
        TNode* n = FindNode(key);
        if (n != Nil) {
            cout << "OK: " << n->Value << endl;
        } else {
            cout << "NoSuchWord\n";
        }
    }

    void VocSave() {
        TString path;
        cin >> path;
        std::ofstream out;
        out.open(path.String(), std::ios_base::binary | std::ios_base::trunc);
        if (out.is_open()) {
            Save(out, Root);
            out.close();
            cout << "OK\n";
        } else {
            cout << "ERROR: couldn't open the output file!\n";
        }
    }

    void VocLoad() {
        TString path;
        cin >> path;
        std::ifstream in;
        in.open(path.String(), std::ios_base::binary);
        if (in.is_open()) {
            Clean(Root);
            Root = Load(in, Nil);
            in.close();
            cout << "OK\n";
        } else {
            cout << "ERROR: couldn't open the input file!\n";
        }
    }

    void Save(std::ofstream& out, TNode* root) {
        if (root == Nil)
            return ;
        size_t keySize = root->Key.StrSize();
        out.write((char*)&keySize, sizeof(keySize));
        out.write(root->Key.String(), keySize);
        out.write((char*)&root->Value, sizeof(root->Value));
        out.write((char*)&root->Color, sizeof(root->Color));
        bool hasLeft = (root->Left != Nil);
        bool hasRight = (root->Right != Nil);
        out.write((char*)&hasLeft, sizeof(hasLeft));
        out.write((char*)&hasRight, sizeof(hasRight));
        if (hasLeft)
            Save(out, root->Left);
        if (hasRight)
            Save(out, root->Right);
    }

    TNode* Load(std::ifstream& in, TNode* par) {
        TNode* n = Nil;
        size_t keySize;
        V val;
        TColor c;
        bool hasLeft = false;
        bool hasRight = false;

        in.read((char*)(&keySize), sizeof(keySize));
        if (in.gcount() == 0)
			return n;
        char* key = new char[keySize + 1];
        key[keySize] = '\0';
        in.read(key, keySize);
        in.read((char*)(&val), sizeof(val));
        in.read((char*)(&c), sizeof(c));
        in.read((char*)(&hasLeft), sizeof(hasLeft));
        in.read((char*)(&hasRight), sizeof(hasRight));

        n = new TNode();
        n->Key.Move(key);
        key = nullptr;
        n->Value = val;
        n->Color = c;
        n->P = par;
        if (hasLeft) {
            n->Left = Load(in, n);
        } else {
            n->Left = Nil;
        }
        if (hasRight) {
            n->Right = Load(in, n);
        } else {
            n->Right = Nil;
        }
        return n;
    }
};

int main() {
    std::ios_base::sync_with_stdio(false);
    std::cin.tie(nullptr);
    TRBTree<TString, unsigned long long> voc;
    TString input;
    TString saveOrLoad;
    while (cin >> input) {
        switch (input[0]) {
            case '+':
                voc.VocAdd();
                break;
            case '-':
                voc.VocErase();
                break;
            case '!':
                cin >> saveOrLoad;
                if (saveOrLoad == "Save")
                    voc.VocSave();
                else if (saveOrLoad == "Load")
                    voc.VocLoad();
                break;
            default:
                voc.VocFind(input);
                break;
        }
    }
    return 0;
}
\end{verbatim}

\subsection*{Тест производительности}

В данных тестах используются строки только из 50 элементов, сгенерированных рандомно. Поэтому возможен случай, что некоторые ключи будут совпадать и не будут добавлены в дерево.

\begin{tikzpicture}
\begin{axis}
\addplot coordinates {
( 100000, 0.059764)
( 200000, 0.109054)
( 400000, 0.218491)};
\end{axis}
\end{tikzpicture}
\newline
Пояснения к графику:
Ось y - время в секундах. Ocь x - количество строк. \newline 

\subsection*{Выводы}

В ходе данной работы я познакомился со структурой данных RBTree. Она представляет из себя самобалансирущееся бинарное дерево поиска. Эта структура довольно экономична по памяти, так как дополнительно каждая вершина хранит только один бит информации (цвет). В среднем случае красно-чёрное дерево будет работать медленне, чем AVL, но при больших объёмах данных оно может быть эффективнее. Также в данной работе был реализован простой бенчмарк, с помощью которого я смог протестировать функцию вставки.

\end{document}
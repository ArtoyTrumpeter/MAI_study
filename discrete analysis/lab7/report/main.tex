\documentclass[12pt]{article}

\usepackage{fullpage}
\usepackage{graphicx}
\usepackage{multicol,multirow}
\usepackage{tabularx}
\usepackage{ulem}
\usepackage{pgfplots}
\usepackage[utf8]{inputenc}
\usepackage[russian]{babel}

\begin{document}
\thispagestyle{empty}
\begin{center}
	\bfseries

	{\Large Московский авиационный институт\\ (национальный исследовательский университет)

	}

	\vspace{48pt}

	{\large Факультет информационных технологий и прикладной математики
	}

	\vspace{36pt}


	{\large Кафедра вычислительной математики и~программирования

	}


	\vspace{48pt}

	{Лабораторная работа №\,7 по курсу дискретного анализа: Динамическое программирование}

\end{center}

\vspace{72pt}

\begin{flushright}
	\begin{tabular}{rl}
		Студент:       & А.\, О. Тояков   \\
		Преподаватель: & А.\, Н. Ридли \\
		Группа:        & М8О-307Б-18      \\
		Дата:          &                 \\
		Оценка:        &                  \\
		Подпись:       &                  \\
	\end{tabular}
\end{flushright}

\vfill

\begin{center}
	\bfseries
	Москва\\
	\the\year
\end{center}

\newpage

\subsection*{Условие}

При помощи метода динамического программирования разработать алгоритм решения задачи, определяемой своим вариантом; оценить время выполнения алгоритма и объем затрачиваемой оперативной памяти. Перед выполнением задания необходимо обосновать применимость метода динамического программирования.
Разработать программу на языке C или C++, реализующую построенный алгоритм.

Задана матрица натуральных чисел A размерности n × m. Из текущей клетки можно перейти в любую из 3-х соседних, стоящих в строке с номером на единицу больше, при этом за каждый проход через клетку (i, j) взымается штраф A(i, j). Необходимо пройти из какой-нибудь клетки верхней строки до любой клетки нижней, набрав при проходе по клеткам минимальный штраф.

\subsection*{Метод решения}

Для реализации поставленной задачи мне понадобится дополнительная структура, куда я буду записывать промежуточные данные.

\begin{enumerate}
\item Создание структуры для хранения промежуточных данных.
\item Проходя по каждому элементу, буду определять минимальное значение штрафа для данного элемента, исходя из предыдущих вычислений.
\item Ещё одним проходом выведу в консоль оптимальное решение.
\end{enumerate}

Таким способом с помощью этого алгоритма я буду считать оптимальный результат для ячеек массива, исходя из результатов подсчёта предыдущих ячеек. Я буду считать сумму, идя от конца данного массива к его началу: данное решение значительно упростит вывод оптимального решения. Идя сверху вниз к текущему элементу B[i, j] - A[i, j], я прибавляю минимальное значение штрафа, выбирая из трёх уже подсчитанных значений min(B[i + 1, j - 1], B[i + 1, j], B[i + 1, j + 1]). Сложность подсчёта O(n * n), а сложность вывода оптимального решения O(2n).

\subsection*{Описание программы}

Для начала я инициализировал 5 переменных типа int, а также создал две матрицы, которые содержат элементы типа long long. Матрица А служит для хранения исходной матрицы, а в матрице B записывается результат вычислений. В данной программе используется 20 байт под статические переменные, а также выделяется 8 * 2 * n байт, где n - размер квадратной матрицы. Также реализована функция min, которая будет вычислять минимальное значение из трёх, поданных на вход. Для элементов, стоящих в начале или в конце строки массива минимум считается для двух элементов. Начальные значения из матрицы B копируются из матрицы А. Оптимальное решение соответствует спуску по матрице B, проходя минимальные значения. 

\subsection*{Исходный код}

\begin{verbatim}
#include <iostream>
#include <string>
#include <iomanip>
#include <sstream>
#include <stdlib.h>
#include <vector>

// ошибки на чекере из-за перевыполнения
long long Minimum(long long first, long long second, long long third) {
    if(first <= second && first <= third) {
        return first;
    } else if(first >= second && second <= third) {
        return second;
    } else if(first >= third && third <= second) {
        return third;
    }
    return 0;
}


int main() {
    // initialization
    long long **A;
    long long **B;
    int n,m;
    std::cin >> n >> m;

    int i = 0, j = 0, k = 0;
    A = new long long*[n];
    for(i = 0; i < n; i++) {
        A[i] = new long long[m];
    }
    for(i = 0; i < n; i++) {
        for(j = 0; j < m;j++) {
            std::cin >> A[i][j]; 
        }
    }
    B = new long long*[n];
    for(i = 0; i < n; i++) {
        B[i] = new long long[m];
    }
    for(i = 0; i < n; i++) {
        for(j = 0; j < m;j++) {
            B[i][j] = 0; 
        }
    }
    // initialization
    // algorithm
    for(j = 0; j < m; j++) {
        B[n - 1][j] = A[n - 1][j];
    }
    for(i = n - 2; i >= 0; i--) {
        B[i][0] = (Minimum(B[i + 1][0], B[i + 1][0], B[i + 1][1]) + A[i][0]);
        for(j = 1; j < m - 1; j++) {
            B[i][j] = (Minimum(B[i + 1][j - 1], B[i + 1][j], B[i + 1][j + 1]) + A[i][j]);
        }
        B[i][m - 1] = (Minimum(B[i + 1][m - 2], B[i + 1][m - 1], B[i + 1][m - 1]) + A[i][m - 1]);
    }
    // algorithm
    // output
    for(j = 1; j < m; j++) {
        if(B[0][k] > B[0][j]) {
            k = j;
        }
    }
    std::cout << B[0][k] << "\n " << "(1," <<  k + 1 << ") ";
    for(i = 1; i < n; i++) {
        if((k > 0) && (B[i][k - 1] < B[i][k])) {
            if((k + 1 < m) && (B[i][k - 1] > B[i][k + 1])) {
                k++;
            } else {
                k--;
            }
        } else {
            if((k + 1 < m) && (B[i][k + 1] < B[i][k])) {
                k++;
            }
        }
        std::cout << "("<< i + 1 << "," <<  k + 1 << ") ";
    }
    std::cout << "\n";
    // output
    // clear
    for(int i = 0; i < n; i++) {
        delete [] A[i];
        delete [] B[i];
    }
    delete [] A;
    delete [] B;
    // clear
    return 0;
}
\end{verbatim}

\subsection*{Тест производительности}

Тесты представляют из матрицы, заполненные рандомными числами не больше 100000000. Для теста были выбраны квадратные матрицы, порядок которых не превышает 1000.

\begin{tikzpicture}
\begin{axis}
\addplot coordinates {
( 100, 0.000216)
( 500, 0.003)
( 1000, 0.01)};
\end{axis}
\end{tikzpicture}
\newline
Пояснения к графику:
Ось y - время в секундах. Ocь x - количество строк.

\subsection*{Выводы}

Для некоторых задач метод рекурсивного разбиения основной задачи на подзадачи является довольно эффективным и полезным. Используя динамическое программирование, можно заметно облегчить процесс решения и уменьшить ассимпотическую сложность. Данная задача была реализована со сложностью O(n * n + 2n). Думаю, что эту же задачу можно решить, не используя методы динамического программирования, а просто, например, перебором, но в этом случае это точно будет менее эффективно и реализация будет довольно сложная.

\end{document}
\documentclass[12pt]{article}

\usepackage{fullpage}
\usepackage{multicol,multirow}
\usepackage{tabularx}
\usepackage{ulem}
\usepackage[utf8]{inputenc}
\usepackage[russian]{babel}
\usepackage{amsmath}
\usepackage{amssymb}

\usepackage{titlesec}

\titleformat{\section}
  {\normalfont\Large\bfseries}{\thesection.}{0.3em}{}

\titleformat{\subsection}
  {\normalfont\large\bfseries}{\thesubsection.}{0.3em}{}

\titlespacing{\section}{0pt}{*2}{*2}
\titlespacing{\subsection}{0pt}{*1}{*1}
\titlespacing{\subsubsection}{0pt}{*0}{*0}
\usepackage{listings}
\lstloadlanguages{Lisp}
\lstset{extendedchars=false,
	breaklines=true,
	breakatwhitespace=true,
	keepspaces = true,
	tabsize=2
}
\begin{document}


\section*{Отчет по лабораторной работе №\,5
по курсу \guillemotleft  Функциональное программирование\guillemotright}
\begin{flushright}
Студент группы 8О-307Б-18 МАИ \textit{Тояков Артем}, \textnumero 22 по списку \\
\makebox[7cm]{Контакты: {\tt temathesuper@mail.ru} \hfill} \\
\makebox[7cm]{Работа выполнена: 14.05.2021 \hfill} \\
\ \\
Преподаватель: Иванов Дмитрий Анатольевич, доц. каф. 806 \\
\makebox[7cm]{Отчет сдан: \hfill} \\
\makebox[7cm]{Итоговая оценка: \hfill} \\
\makebox[7cm]{Подпись преподавателя: \hfill} \\

\end{flushright}

\section{Тема работы}
Обобщённые функции, методы и классы объектов.

\section{Цель работы}
Изучить обобщённые функции, методы и классы объектов, а также методы работы с ними в Коммон Лисп.

\section{Задание (вариант № 5.23)}
Задание: Определить обобщённую функцию и методы on-single-line3-p - предикат, принимающий в качестве аргументов три точки (радиус-вектора) и необязательный параметр tolerance (допуск), возвращающий T, если три указанные точки лежат на одной прямой (вычислять с допустимым отклонением tolerance).

Точки могут быть заданы как декартовыми координатами (экземплярами cart), так и полярными (экземплярами polar).

\section{Оборудование студента}
Процессор: Intel(R) Core(TM) i7-8565U CPU @ 1.80GHz, память: 3,8 Gb, разрядность системы: 64.

\section{Программное обеспечение}
UBUNTU 18.04.5 LTS, компилятор sbcl

\section{Идея, метод, алгоритм}

Идея в том, чтобы воспользоваться уравнением прямой, проходящей через 2 точки и подставить вместо x, y координаты третьей точки и подсчитать выражение, и если эта разность по модулю будет <= введённого значения tolerance, то условие выполняется и выводится T. Также стоит прописать отдельные методы для представления точек в декартовых и полярных координтах.

\section{Сценарий выполнения работы}
\begin{itemize}
\setlength{\itemsep}{-1mm}
\item Анализ возможных реализаций поставленной задачи на Коммон Лисп
\item Изучение синтаксиса и основных функций работы с обобщёнными фукнциями и классами Коммон Лисп
\item Реализация поставленной задачи на Коммон Лисп
\end{itemize}
\section{Распечатка программы и её результаты}

\subsection{Исходный код}
\begin{verbatim}

(defun square (x) (* x x))

(defclass cart ()                ; имя класса и надклассы
  ((x :initarg :x :reader cart-x)   ; дескриптор слота x
   (y :initarg :y :reader cart-y))) ; дескриптор слота y


(defmethod print-object ((c cart) stream)
    (format stream "[CART x:~d y:~d]"
        (cart-x c) (cart-y c)
    )
)

(defclass polar ()
  ((radius :initarg :radius :accessor radius) 	; длина >=0
   (angle  :initarg :angle  :accessor angle)))	; угол (-pi;pi]


(defmethod print-object ((p polar) stream)
  (format stream "[POLAR radius ~d angle ~d]"
          (radius p) (angle p)))


(defmethod radius ((c cart))
  (sqrt (+ (square (cart-x c))
           (square (cart-y c)))))


(defmethod angle ((c cart))
  (atan (cart-y c) (cart-x c)))	; atan2 в Си


(defmethod cart-x ((p polar))
  (* (radius p) (cos (angle p))))


(defmethod cart-y ((p polar))
  (* (radius p) (sin (angle p))))

(defun on-single-line3 (data)
	(<= (abs (- (* (- (cart-x (second data)) (cart-x (first data)))
	   	   (- (cart-y (third data)) (cart-y (first data))))
	       (* (- (cart-x (third data)) (cart-x (first data)))
	       (- (cart-y (second data)) (cart-y (first data)))))
	    ) (fourth data)
	)
)

(defgeneric on-single-line3-p (v1 v2 v3 &optional tolerance))

(defmethod on-single-line3-p ((v1 cart) (v2 cart) (v3 cart) &optional (tolerance 0.001))  
	(on-single-line3 (list v1 v2 v3 tolerance)))

(defmethod on-single-line3-p ((v1 polar) (v2 polar) (v3 polar) &optional (tolerance 0.001))  
	(on-single-line3 (list v1 v2 v3 tolerance)))

\end{verbatim}

\subsection{Результаты работы}
\begin{verbatim}
* (setq v1 (make-instance 'cart :x 0 :y 0))

[CART x:0 y:0]
* (setq v2 (make-instance 'cart :x 1 :y 1))

[CART x:1 y:1]
* (setq v3 (make-instance 'cart :x 2 :y 2))

[CART x:2 y:2]
* (on-single-line3-p (v1 v2 v3 0.0001))

T
* (setq v4 (make-instance 'polar :radius 1 :angle (/ pi 4)))

[POLAR radius 1 angle 0.7853981633974483d0]
* (setq v5 (make-instance 'polar :radius 0 :angle 0))

[POLAR radius 0 angle 0]
* (setq v6 (make-instance 'polar :radius 0 :angle 2))

[POLAR radius 0 angle 2]
* (on-single-line3-p (v4 v5 v6))

NIL
\end{verbatim}

\section{Дневник отладки}
\begin{tabular}{|p{50pt}|p{130pt}|p{130pt}|p{70pt}|}
\hline
Дата & Событие & Действие по исправлению & Примечание \\ \hline
& & &\\
\hline
\end{tabular}

\section{Замечания автора по существу работы}

\section{Выводы}
В ходе данной работы мне удалось познакомиться с обобщёнными функциями, методами и классами в Коммон Лисп. Лисп был создан за пару десятилетий до того момента, когда объектно-ориентированное программирование стало популярным. Хотел бы отметить, что Коммон Лисп является полноценным объектно-ориентированным языком и в данном языке заложены основные парадигмы/принципы объектно-ориентированного программирования.

\end{document}
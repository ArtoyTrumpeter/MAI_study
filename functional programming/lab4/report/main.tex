\documentclass[12pt]{article}

\usepackage{fullpage}
\usepackage{multicol,multirow}
\usepackage{tabularx}
\usepackage{ulem}
\usepackage[utf8]{inputenc}
\usepackage[russian]{babel}
\usepackage{amsmath}
\usepackage{amssymb}

\usepackage{titlesec}

\titleformat{\section}
  {\normalfont\Large\bfseries}{\thesection.}{0.3em}{}

\titleformat{\subsection}
  {\normalfont\large\bfseries}{\thesubsection.}{0.3em}{}

\titlespacing{\section}{0pt}{*2}{*2}
\titlespacing{\subsection}{0pt}{*1}{*1}
\titlespacing{\subsubsection}{0pt}{*0}{*0}
\usepackage{listings}
\lstloadlanguages{Lisp}
\lstset{extendedchars=false,
	breaklines=true,
	breakatwhitespace=true,
	keepspaces = true,
	tabsize=2
}
\begin{document}


\section*{Отчет по лабораторной работе №\,4 
по курсу \guillemotleft  Функциональное программирование\guillemotright}
\begin{flushright}
Студент группы 8О-307Б-18 МАИ \textit{Тояков Артем}, \textnumero 22 по списку \\
\makebox[7cm]{Контакты: {\tt temathesuper@mail.ru} \hfill} \\
\makebox[7cm]{Работа выполнена: 23.04.2021 \hfill} \\
\ \\
Преподаватель: Иванов Дмитрий Анатольевич, доц. каф. 806 \\
\makebox[7cm]{Отчет сдан: \hfill} \\
\makebox[7cm]{Итоговая оценка: \hfill} \\
\makebox[7cm]{Подпись преподавателя: \hfill} \\

\end{flushright}

\section{Тема работы}
Знаки и строки.

\section{Цель работы}
Изучить знаки и строки, а также методы работы с ними в Коммон Лисп.

\section{Задание (вариант № 4.11)}
Запрограммировать на языке Коммон Лисп функцию, принимающую один аргумент - предложение. Функция должна возвращать число слов в этом предложении, у которых первый и последний знак совпадают. Сравнение как латинских букв, так и русских должно быть регистро-независимым.

\section{Оборудование студента}
Процессор: Intel(R) Core(TM) i7-8565U CPU @ 1.80GHz, память: 3,8 Gb, разрядность системы: 64.

\section{Программное обеспечение}
UBUNTU 18.04.5 LTS, компилятор sbcl

\section{Идея, метод, алгоритм}
Идея в том, чтобы пройти по всему предложению, разделяя его по словам, сравнивать первую и последнюю буквы всех слов, и в тех случаях, где они равны, увеличивать счётчик на 1. В конце вывести результат - переменную счётчик.
\\
В программе есть одна основная функция (defun count-words-with-start-eq-end(str)), в которой в начале объявляются переменные, а затем с помощью цикла loop и двух итераторов i, j реализовано разбиение на слова и сравнение букв. Один проход по внешнему циклу - проверка одного слова.  

\section{Сценарий выполнения работы}
\begin{itemize}
\setlength{\itemsep}{-1mm}
\item Анализ возможных реализаций поставленной задачи на Коммон Лисп
\item Изучение синтаксиса и основных функций работы со знаками и строками Коммон Лисп
\item Реализация поставленной задачи на Коммон Лисп
\end{itemize}
\section{Распечатка программы и её результаты}

\subsection{Исходный код}
\begin{verbatim}
(defun whitespace-char(ch)
	(member ch '(#\Space #\Tab #\Newline))
)

(defun russian-upper-case-p (char)
    (position char "АБВГДЕЁЖЗИЙКЛМНОПРСТУФХЦЧШЩЪЫЬЭЮЯ")
)

(defun russian-char-downcase (char)
	(let ((i (russian-upper-case-p char)))
        (if i 
            (char "абвгдеёжзийклмнопрстуфхцчшщъыьэюя" i)
            (char-downcase char)
        )
    )
)  

(defun russian-char-equal (char1 char2)
    (char-equal (russian-char-downcase char1)
        (russian-char-downcase char2))
)

(defun endSentence-char(ch)
	(member ch '(#\! #\? #\. #\"))
)

(defun count-words-with-start-eq-end(str)
	(let
		(
			(res 0)
            (i 0)
            (j 0)
			(cur-ch-begin nil)
			(cur-ch-end nil)
		)
		
		(loop while (and (not (endSentence-char (char str i))) (< i (length str))) do
			(setq cur-ch-begin (char str i))
			(setq j i)
            (loop while (and (< j (length str)) (and (not (whitespace-char (char str j))) (not (endSentence-char (char str j))))) do
                (setq j (+ j 1))
            )
            (setq j (- j 1))
            (setq cur-ch-end (char str j))
			(if (or (char-equal cur-ch-begin cur-ch-end) (russian-char-equal cur-ch-begin cur-ch-end))
				(setq res (+ res 1))
			)
			(setq i (+ j 1))
            (loop while (and (< i (length str)) (and (not (endSentence-char (char str i))) (whitespace-char (char str i)))) do
                (setq i (+ i 1))
            )
		)
		(write res)
	)
)
\end{verbatim}

\subsection{Результаты работы}
\begin{verbatim}
* (count-words-with-start-eq-end "а роза упала на лапу Азора")
2
2
* (count-words-with-start-eq-end "hepl fuf me")
1
1
* (count-words-with-start-eq-end "К долинам,    я крутой!") 
2
2
* (count-words-with-start-eq-end "Двойные кавычки активно используются в русском языке в машинном тексте!?")
3
3
* (count-words-with-start-eq-end "Ала ара как лел мор мом троп?")
5
5
* (count-words-with-start-eq-end "Америка.")
1
1
* (count-words-with-start-eq-end "На столе вкусная айва или баба пришла, где арка.")
3
3
* (count-words-with-start-eq-end "Hello, world ahaha.")
1
1
\end{verbatim}

\section{Дневник отладки}
\begin{tabular}{|p{50pt}|p{130pt}|p{130pt}|p{70pt}|}
\hline
Дата & Событие & Действие по исправлению & Примечание \\ \hline
& & &\\
\hline
\end{tabular}

\section{Замечания автора по существу работы}
Единственный недочёт: результат выводится 2 раза.

\section{Выводы}
В ходе данной работы мне удалось познакомиться со встроенными функциями/инструментами для работы со знаками и строками. Со строками я был знаком и ранее, однако было довольно интересно увидеть применение такой структуры данных в Коммон Лисп. В моей программе алгоритм работает за линейное время O(n), где n - длина исследуемого предложения.

\end{document}
\documentclass[12pt]{article}

\usepackage{fullpage}
\usepackage{multicol,multirow}
\usepackage{tabularx}
\usepackage{ulem}
\usepackage[utf8]{inputenc}
\usepackage[russian]{babel}
\usepackage{amsmath}
\usepackage{amssymb}

\usepackage{titlesec}

\titleformat{\section}
  {\normalfont\Large\bfseries}{\thesection.}{0.3em}{}

\titleformat{\subsection}
  {\normalfont\large\bfseries}{\thesubsection.}{0.3em}{}

\titlespacing{\section}{0pt}{*2}{*2}
\titlespacing{\subsection}{0pt}{*1}{*1}
\titlespacing{\subsubsection}{0pt}{*0}{*0}
\usepackage{listings}
\lstloadlanguages{Lisp}
\lstset{extendedchars=false,
	breaklines=true,
	breakatwhitespace=true,
	keepspaces = true,
	tabsize=2
}
\begin{document}


\section*{Отчет по лабораторной работе №\,4 
по курсу \guillemotleft  Функциональное программирование\guillemotright}
\begin{flushright}
Студент группы 8О-307Б-18 МАИ \textit{Тояков Артем}, \textnumero 22 по списку \\
\makebox[7cm]{Контакты: {\tt temathesuper@mail.ru} \hfill} \\
\makebox[7cm]{Работа выполнена: 23.04.2021 \hfill} \\
\ \\
Преподаватель: Иванов Дмитрий Анатольевич, доц. каф. 806 \\
\makebox[7cm]{Отчет сдан: \hfill} \\
\makebox[7cm]{Итоговая оценка: \hfill} \\
\makebox[7cm]{Подпись преподавателя: \hfill} \\

\end{flushright}

\section{Тема работы}
Знаки и строки.

\section{Цель работы}
Изучить знаки и строки, а также методы работы с ними в Коммон Лисп.

\section{Задание (вариант № 4.11)}
Запрограммировать на языке Коммон Лисп функцию, принимающую один аргумент - предложение. Функция должна возвращать число слов в этом предложении, у которых первый и последний знак совпадают. Сравнение как латинских букв, так и русских должно быть регистро-независимым.

\section{Оборудование студента}
Процессор: Intel(R) Core(TM) i7-8565U CPU @ 1.80GHz, память: 3,8 Gb, разрядность системы: 64.

\section{Программное обеспечение}
UBUNTU 18.04.5 LTS, компилятор sbcl

\section{Идея, метод, алгоритм}
Идея в том, чтобы пройти по всему предложению, разделяя его по словам, сравнивать первую и последнюю буквы всех слов, и в тех случаях, где они равны, увеличивать счётчик на 1. В конце вывести результат - переменную счётчик.
\\
В программе есть одна основная функция (defun count-words-with-start-eq-end (txt)), в которой в начале объявляются переменные, а затем с помощью дополнительной функции word-list реализовано разбиение на слова.

\section{Сценарий выполнения работы}
\begin{itemize}
\setlength{\itemsep}{-1mm}
\item Анализ возможных реализаций поставленной задачи на Коммон Лисп
\item Изучение синтаксиса и основных функций работы со знаками и строками Коммон Лисп
\item Реализация поставленной задачи на Коммон Лисп
\end{itemize}
\section{Распечатка программы и её результаты}

\subsection{Исходный код}
\begin{verbatim}

(defun whitespace-char-p (char)
    (member char '(#\Space #\Tab #\Newline)))

(defun word-list (string)
    (loop with len = (length string)
        for left = 0 then (1+ right)
        for right = (or (position-if #'whitespace-char-p string :start left) len)
        unless (= right left)
            collect (subseq string left right)
        while (< right len)
    )
)

(defun russian-upper-case-p (char)
    (position char "АБВГДЕЁЖЗИЙКЛМНОПРСТУФХЦЧШЩЪЫЬЭЮЯ")
)

(defun russian-char-downcase (char)
    (let ((i (russian-upper-case-p char)))
        (if i 
            (char "абвгдеёжзийклмнопрстуфхцчшщъыьэюя" i)
            (char-downcase char)
        )
    )
)

(defun russian-char-equal (char1 char2)
    (char-equal (russian-char-downcase char1)
        (russian-char-downcase char2)
    )
)

(defun english-upper-case-p (char)
    (position char "ABCDEFGHIGKLMNOPQRSTUVWXYZ")
)

(defun english-char-downcase (char)
    (let ((i (english-upper-case-p char)))
        (if i 
            (char "abcdefghigklmnopqrstuvwxyz" i)
            (char-downcase char)
        )
    )
)

(defun english-char-equal (char1 char2)
    (char-equal (english-char-downcase char1)
        (english-char-downcase char2)
    )
)

(defun count-words-with-start-eq-end (txt)
    (let ((found 0))
    (dolist (sentence txt)
        (dolist (word (word-list sentence))
            (let ((first-char NIL) (last-char NIL))
            (setf first-char (char word 0))
            (setf last-char (char word (- (length word) 1)))
                
            (let ((rus-res(russian-char-equal first-char last-char)) (eng-res(english-char-equal first-char last-char)))
                (if (or rus-res eng-res) (setf found (+ found 1)))
            )
            )
        )
    )
    found)
)

\end{verbatim}

\subsection{Результаты работы}
\begin{verbatim}
* (count-words-with-start-eq-end '("а роза упала на лапу Азора"))

2
* (count-words-with-start-eq-end '("Ала ара как лел мор мом троп?"))

5
* (count-words-with-start-eq-end '("Двойные кавычки активно используются в русском языке в машинном тексте!?"))

3
* (count-words-with-start-eq-end '("Hello, world ahah."))

0
\end{verbatim}

\section{Дневник отладки}
\begin{tabular}{|p{50pt}|p{130pt}|p{130pt}|p{70pt}|}
\hline
Дата & Событие & Действие по исправлению & Примечание \\ \hline
06.05.2021 & Ошибка: выход за границу памяти массива & Довольно сильное изменение программы, ввод структуры данных: list &\\
\hline
\end{tabular}

\section{Замечания автора по существу работы}

\section{Выводы}
В ходе данной работы мне удалось познакомиться со встроенными функциями/инструментами для работы со знаками и строками. Со строками я был знаком и ранее, однако было довольно интересно увидеть применение такой структуры данных в Коммон Лисп. В моей программе алгоритм работает за линейное время O(n), где n - длина исследуемого предложения.

\end{document}
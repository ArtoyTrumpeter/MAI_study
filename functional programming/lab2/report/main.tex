\documentclass[12pt]{article}

\usepackage{fullpage}
\usepackage{multicol,multirow}
\usepackage{tabularx}
\usepackage{ulem}
\usepackage[utf8]{inputenc}
\usepackage[russian]{babel}
\usepackage{amsmath}
\usepackage{amssymb}

\usepackage{titlesec}

\titleformat{\section}
  {\normalfont\Large\bfseries}{\thesection.}{0.3em}{}

\titleformat{\subsection}
  {\normalfont\large\bfseries}{\thesubsection.}{0.3em}{}

\titlespacing{\section}{0pt}{*2}{*2}
\titlespacing{\subsection}{0pt}{*1}{*1}
\titlespacing{\subsubsection}{0pt}{*0}{*0}
\usepackage{listings}
\lstloadlanguages{Lisp}
\lstset{extendedchars=false,
	breaklines=true,
	breakatwhitespace=true,
	keepspaces = true,
	tabsize=2
}
\begin{document}


\section*{Отчет по лабораторной работе №\,2 
по курсу \guillemotleft  Функциональное программирование\guillemotright}
\begin{flushright}
Студент группы 8О-307Б-18 МАИ \textit{Тояков Артем}, \textnumero 22 по списку \\
\makebox[7cm]{Контакты: {\tt temathesuper@mail.ru} \hfill} \\
\makebox[7cm]{Работа выполнена: 09.04.2021 \hfill} \\
\ \\
Преподаватель: Иванов Дмитрий Анатольевич, доц. каф. 806 \\
\makebox[7cm]{Отчет сдан: \hfill} \\
\makebox[7cm]{Итоговая оценка: \hfill} \\
\makebox[7cm]{Подпись преподавателя: \hfill} \\

\end{flushright}

\section{Тема работы}
Простейшие функции работы со списками Коммон Лисп.

\section{Цель работы}
Изучить  простейшие функции работы со списками Коммон Лисп.

\section{Задание (вариант № 2.17)}
Запрограммируйте на языке Коммон Лисп функцию, которая принимает в качестве аргументов натуральное число i и список lst и выдает элемент списка, имеющий индекс (номер) i, индексация с 0. Если i больше или равно длине lst, то вернуть NIL.

\section{Оборудование студента}
Процессор: Intel(R) Core(TM) i7-8565U CPU @ 1.80GHz, память: 3,8 Gb, разрядность системы: 64.

\section{Программное обеспечение}
UBUNTU 18.04.5 LTS, компилятор sbcl

\section{Идея, метод, алгоритм}
Идея в том, чтобы сравнивать индекс с 0 и уменьшать его на 1, пока он не станет 0. Тогда мы выведем требующийся элемент списка.
\begin{itemize}
\setlength{\itemsep}{-1mm} % уменьшает расстояние между элементами списка
\item (nth-element (i list)) - вызов функции find-el, с предварительной проверкой корректности индекса с помощью условного оператора.
\item (find-el (i list)) - рекурсивная функция: с помощью предиката cond мы проверяем словие с предикатом zerop, который, в свою очередь проверяет является ли нулевым наш индекс и если да, то мы с помощью предиката car выводим нужный элемент. Если же нет функция вызывается рекурсивно от индекса уменьшенного на 1 и списка, у которого удаляется текущий первый элемент с помощью предиката cdr.
\end{itemize}

\section{Сценарий выполнения работы}
\begin{itemize}
\setlength{\itemsep}{-1mm}
\item Анализ возможных реализаций поставленной задачи на common Lisp
\item Изучение синтаксиса и основных функций работы со списками common Lisp
\item Реализация поставленной задачи на common Lisp
\item Сравнение результатов работы написанной программы и стандартного предиката nth
\end{itemize}
\section{Распечатка программы и её результаты}

\subsection{Исходный код}
\begin{verbatim}
(defun nth-element (i list)
    (if (< i (length list))
        (find-el i list)
    )
)

(defun find-el (index list)
    (cond
        ((zerop index) (car list))
        (t (find-el (- index 1) (cdr list)))
    )
)
\end{verbatim}

\subsection{Результаты работы}
\begin{verbatim}
* (nth-element 3 '(a0 b1 c2 d3 e4))

D3
* (nth-element 5 '(a0 b1 c2 d3 e4)) 

NIL
* (nth-element 4 '(ubefbew ejfefn fenfie ijijaoo fejfefeihf nennn))

FEJFEFEIHF

* (nth-element 0 '(1 2 3 2343 3242))

1
* (nth-element 6 '(7df7 s7fy ew8 wbf3 23jb2 uhf9 we2))

WE2

* (nth 3 '(a0 b1 c2 d3 e4))

D3
* (nth 5 '(a0 b1 c2 d3 e4))

NIL
* (nth 4 '(ubefbew ejfefn fenfie ijijaoo fejfefeihf nennn))

FEJFEFEIHF

* (nth 0 '(1 2 3 2343 3242))

1
* (nth 6 '(7df7 s7fy ew8 wbf3 23jb2 uhf9 we2))

WE2
\end{verbatim}

\section{Дневник отладки}
\begin{tabular}{|p{50pt}|p{130pt}|p{130pt}|p{70pt}|}
\hline
Дата & Событие & Действие по исправлению & Примечание \\ \hline
22.04.2021 & Некорректное использование (atom 1) & Замена этого выражения на t в операторе cond &\\ \hline
22.04.2021 & Неэффективность: вызов list-length & Замена этой функции на length & Без этого не обойтись, т. к. иначе не удастся наложить ограничения на i \\
\hline
\end{tabular}

\section{Замечания автора по существу работы}
Замечаний нет.

\section{Выводы}
В ходе данной работы я познакомился с представлением и основными особенностями списков в Коммон Лисп. Список в Коммон Лисп представляет собой S-выражение вида: атом | список(хвост). В конце работы я сравнил результат работы моей программы и стандартного предиката nth и после того, как они вывели идентичные значения для одинаковых входных данных, я убедился в том, что алгоритм реализован корректно.

\end{document}
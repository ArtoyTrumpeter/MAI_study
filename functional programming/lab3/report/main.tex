\documentclass[12pt]{article}

\usepackage{fullpage}
\usepackage{multicol,multirow}
\usepackage{tabularx}
\usepackage{ulem}
\usepackage[utf8]{inputenc}
\usepackage[russian]{babel}
\usepackage{amsmath}
\usepackage{amssymb}

\usepackage{titlesec}

\titleformat{\section}
  {\normalfont\Large\bfseries}{\thesection.}{0.3em}{}

\titleformat{\subsection}
  {\normalfont\large\bfseries}{\thesubsection.}{0.3em}{}

\titlespacing{\section}{0pt}{*2}{*2}
\titlespacing{\subsection}{0pt}{*1}{*1}
\titlespacing{\subsubsection}{0pt}{*0}{*0}
\usepackage{listings}
\lstloadlanguages{Lisp}
\lstset{extendedchars=false,
	breaklines=true,
	breakatwhitespace=true,
	keepspaces = true,
	tabsize=2
}
\begin{document}


\section*{Отчет по лабораторной работе №\,3 
по курсу \guillemotleft  Функциональное программирование\guillemotright}
\begin{flushright}
Студент группы 8О-307Б-18 МАИ \textit{Тояков Артем}, \textnumero 22 по списку \\
\makebox[7cm]{Контакты: {\tt temathesuper@mail.ru} \hfill} \\
\makebox[7cm]{Работа выполнена: 12.04.2021 \hfill} \\
\ \\
Преподаватель: Иванов Дмитрий Анатольевич, доц. каф. 806 \\
\makebox[7cm]{Отчет сдан: \hfill} \\
\makebox[7cm]{Итоговая оценка: \hfill} \\
\makebox[7cm]{Подпись преподавателя: \hfill} \\

\end{flushright}

\section{Тема работы}
Последовательности, массивы и управляющие конструкции Коммон Лисп.

\section{Цель работы}
Изучить Последовательности, массивы и управляющие конструкции Коммон Лисп.

\section{Задание (вариант № 3.17)}
Запрограммировать на языке Коммон Лисп функцию, принимающую три аргумента:
\begin{itemize}
    \item A - двумерный массив, представляющий действительную матрицу размера m×n,
    \item u - вектор действительных чисел длины n,
    \item i - номер строки, 0 $\leqslant$ i $\leqslant$ m.
\end{itemize}
    
Функция должна возвращать новую матрицу размера (m+1)×n, полученную вставкой после строки с номером i новой строки с элементами из u. i=0 означает вставку перед первой строкой.

Исходный массив A должен оставаться неизменным.

\section{Оборудование студента}
Процессор: Intel(R) Core(TM) i7-8565U CPU @ 1.80GHz, память: 3,8 Gb, разрядность системы: 64.

\section{Программное обеспечение}
UBUNTU 18.04.5 LTS, компилятор sbcl

\section{Идея, метод, алгоритм}
Метод состоит из двух итераций. Идея в том, чтобы для начала вставить u-вектор на место i+1 строки матрицы b. Затем же в полученную матрицу на место j+1 столбца вставить v-вектор. Мне предстоит разбить матрицу на составные части и по новому собрать ее. В программе две основные функции:
\begin{itemize}
\setlength{\itemsep}{-1mm} % уменьшает расстояние между элементами списка
\item (line-extended-matrix a k v) - В данной функции происходит создание матрицы b и копирование элементов матрицы до элемента a[k][...]. Затем же вставка вектора строки на позицию b[k][...] и копирование остальных элементов, если оно необходимо.
\item (extended-matrix a u i) - Функция-вызов для функции line-extended-matrix, где учитываются ограничения, наложенные на индекс i.
\end{itemize}

\section{Сценарий выполнения работы}
\begin{itemize}
\setlength{\itemsep}{-1mm}
\item Анализ возможных реализаций поставленной задачи на common Lisp
\item Изучение синтаксиса и основных функций работы со списками common Lisp
\item Реализация поставленной задачи на common Lisp
\end{itemize}

\section{Распечатка программы и её результаты}

\subsection{Исходный код}
\begin{verbatim}
(defun line-extended-matrix (a k v)
    (let ((m (array-dimension a 0))
        (n (array-dimension a 1)))
        (let ((b (make-array (list (+ 1 m) n)))) ; b = Matrix(* (1 + m) n)
            (dotimes (i k)
                (dotimes (j n)
                    (setf (aref b i j) (aref a i j)))) ; b[i][j] = a[i][j]
            (loop
                :for j :below n
                :do (setf (aref b k j) (aref v j))) ; b[k][j] = v[j], j : 0 to n
            (dotimes (j n)
                (loop :for i :from k :to (- m 1) :do
                    (setf (aref b (+ 1 i) j) (aref a i j)))) ; b[i + 1][j] = a[i][j]
        b)
    )
)

(defun extended-matrix (a u i)
    (array-dimension a 0)
    (array-dimension a 1)
    (if (and (<= i (array-dimension a 0)) (>= i 0)) (line-extended-matrix a i u))
)
\end{verbatim}

\subsection{Результаты работы}
Файл с тестовыми переменными прикреплён к работе.
\begin{verbatim}
* (extended-matrix m1 u1 0)

#2A((1 1 1) (0 0 0) (0 0 0) (0 0 0))
* (extended-matrix m2 u2 1)

#2A((0 0 0 0) (1 1 1 1) (0 0 0 0))
* (extended-matrix m3 u3 2)

#2A((0 0 0 0) (0 0 0 0) (1 1 1 1) (0 0 0 0))
* (extended-matrix m4 u4 0)

#2A((1) (0))
* (extended-matrix m4 u4 1)

#2A((0) (1))
* (extended-matrix m4 u4 2)

NIL
* (extended-matrix m5 u5 0)

#2A((1 1) (0 0) (0 0) (0 0) (0 0))
* (extended-matrix m5 u5 4)

#2A((0 0) (0 0) (0 0) (0 0) (1 1))
\end{verbatim}

\section{Дневник отладки}
\begin{tabular}{|p{50pt}|p{130pt}|p{130pt}|p{70pt}|}
\hline
Дата & Событие & Действие по исправлению & Примечание \\ \hline
& & &\\
\hline
\end{tabular}

\section{Замечания автора по существу работы}
Довольно долго пришлось разбираться с управляющими конструкциями в Common Lisp.

\section{Выводы}
В ходе данной работы мне удалось познакомиться со встроенными функциями/инструментами, а также управляющими конструкциями коммон лисп с помощью которых мне удалось реализовать классический обход по матрице, используя циклы. Это поможет мне легче понимать, как работает язык и облегчит будущую работу с массивами и матрицами.

\end{document}
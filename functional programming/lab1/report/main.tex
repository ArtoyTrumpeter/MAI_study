\documentclass[12pt]{article}

\usepackage{fullpage}
\usepackage{multicol,multirow}
\usepackage{tabularx}
\usepackage{ulem}
\usepackage[utf8]{inputenc}
\usepackage[russian]{babel}
\usepackage{amsmath}
\usepackage{amssymb}

\usepackage{titlesec}

\titleformat{\section}
  {\normalfont\Large\bfseries}{\thesection.}{0.3em}{}

\titleformat{\subsection}
  {\normalfont\large\bfseries}{\thesubsection.}{0.3em}{}

\titlespacing{\section}{0pt}{*2}{*2}
\titlespacing{\subsection}{0pt}{*1}{*1}
\titlespacing{\subsubsection}{0pt}{*0}{*0}
\usepackage{listings}
\lstloadlanguages{Lisp}
\lstset{extendedchars=false,
	breaklines=true,
	breakatwhitespace=true,
	keepspaces = true,
	tabsize=2
}
\begin{document}


\section*{Отчет по лабораторной работе №\,1 
по курсу \guillemotleft  Функциональное программирование\guillemotright}
\begin{flushright}
Студент группы 8О-307Б-18 МАИ \textit{Тояков Артем}, \textnumero 22 по списку \\
\makebox[7cm]{Контакты: {\tt temathesuper@mail.ru} \hfill} \\
\makebox[7cm]{Работа выполнена: 08.04.2021 \hfill} \\
\ \\
Преподаватель: Иванов Дмитрий Анатольевич, доц. каф. 806 \\
\makebox[7cm]{Отчет сдан: \hfill} \\
\makebox[7cm]{Итоговая оценка: \hfill} \\
\makebox[7cm]{Подпись преподавателя: \hfill} \\

\end{flushright}

\section{Тема работы}
Примитивные функции и особые операторы в Common Lisp.

\section{Цель работы}
Изучить примитивные функции и особые операторы в Common Lisp.

\section{Задание (вариант № 1.18 )}
Запрограммируйте на языке Коммон Лисп функцию-предикат с тремя параметрами - действительными положительными числами a, b, c. Функция должна возвращать T (истину), если существует треугольник с длинами сторон a, b и c.

\section{Оборудование студента}
Процессор: Intel(R) Core(TM) i7-8565U CPU @ 1.80GHz, память: 3,8 Gb, разрядность системы: 64.

\section{Программное обеспечение}
UBUNTU 18.04.5 LTS, компилятор sbcl

\section{Идея, метод, алгоритм}
Алгоритм будет применён для всех возможных комбинаций сторон (всего 3, т. е. c > a + b, a > c + b, b > a + c). Идея в том, чтобы сравнить длину стороны с суммой длин двух оставшихся, и если первая величина окажется больше, то треугольник существовать не может, то есть мы будем идти от обратного. В данной программе реализованы одна ключевая функция (triangle-p (a b c)). В ней с помощью предиката cond мы поочереди проверяем 3 наших условия, и как только одно из них окажется true мы вернём Nil. Иначе предикат проверит последнее условие - проверку на атомарность и выведет T.

\section{Сценарий выполнения работы}
\begin{itemize}
\setlength{\itemsep}{-1mm}
\item Анализ возможных реализаций поставленной задачи на common Lisp
\item Изучение синтаксиса и основных функций common Lisp
\item Реализация поставленной задачи на common Lisp
\end{itemize}
\section{Распечатка программы и её результаты}

\subsection{Исходный код}
\begin{verbatim}
(defun triangle-p (a b c)
    (cond
        ((> c (+ a b)) Nil)
        ((> a (+ b c)) Nil)
        ((> b (+ a c)) Nil)
        ((atom 1) T)
    )
)
\end{verbatim}
%\lstinputlisting{./lab2.lisp}

\subsection{Результаты работы}
\begin{verbatim}
* (triangle-p 5.3 3.1 4.0)

T
* (triangle-p 10.3 3.1 4.0)

NIL
* (triangle-p 84.4224 96.343434 78.934434)

T
* (triangle-p 3232.34324 3434.2222 5641.233)

T
\end{verbatim}
%\lstinputlisting{./log2.lisp}

\section{Дневник отладки}
\begin{tabular}{|p{50pt}|p{130pt}|p{130pt}|p{70pt}|}
\hline
Дата & Событие & Действие по исправлению & Примечание \\ \hline
 &  &  &\\
\hline
\end{tabular}

\section{Замечания автора по существу работы}
Программа будет работать некорректно только в одном случае: если на вход подать нулевую сторону.

\section{Выводы}
Данная работа позволила отойти от стандартного императивного программирования и взглянуть на решение поставленной задачи  в функциональной парадигме. Данный алгоритм тривиален и работает за константное время. Также хотелось бы отметить, что в ходе данной работы познакомился с синтаксисом common Lisp и некоторыми основными функциями common Lisp.

\end{document}